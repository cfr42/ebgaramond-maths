% \iffalse meta-comment
%%%%%%%%%%%%%%%%%%%%%%%%%%%%%%%%%%%%%%%%%%%%%%%%%
% ebgaramond-maths.dtx
% Additions and changes Copyright (C) 2014-2025 Clea F. Rees.
% Code from skeleton.dtx Copyright (C) 2015-2024 Scott Pakin (see below).
%
% This work may be distributed and/or modified under the
% conditions of the LaTeX Project Public License, either version 1.3c
% of this license or (at your option) any later version.
% The latest version of this license is in
%   https://www.latex-project.org/lppl.txt
% and version 1.3c or later is part of all distributions of LaTeX
% version 2008-05-04 or later.
%
% This work has the LPPL maintenance status 'muaintained'.
%
% The Current Maintainer of this work is Clea F. Rees.
%
% This work consists of all files listed in manifest.txt.
%
% The file ebgaramond-maths.dtx is a derived work under the terms of the
% LPPL. It is based on version 2.4 of skeleton.dtx which is part of
% dtxtut by Scott Pakin. A copy of dtxtut, including the
% unmodified version of skeleton.dtx is available from
% https://www.ctan.org/pkg/dtxtut and released under the LPPL.
%%%%%%%%%%%%%%%%%%%%%%%%%%%%%%%%%%%%%%%%%%%%%%%%%
% \fi
%
% \iffalse
%<*driver>
\RequirePackage{svn-prov}
% ref. ateb Max Chernoff: https://tex.stackexchange.com/a/723294/
\def\MakePrivateLetters{\makeatletter\ExplSyntaxOn\endlinechar13}
\ExplSyntaxOff
\ProvidesFileSVN{$Id: ebgaramond-maths.dtx 10780 2025-02-07 08:15:26Z cfrees $}[v0.0 \revinfo][\filename: \filebase]
\revinfo][\filebase DTX: ]
\DefineFileInfoSVN[ebgaramond-maths]
\documentclass[11pt,british]{ltxdoc}
% l3doc loads fancyvrb
% fancyvrb overwrites svn-prov's macros without warning
% restore \fileversion \filerev in case we're using l3doc
\GetFileInfoSVN{ebgaramond-maths}
\usepackage[T1]{fontenc}
\usepackage{upquote}
\usepackage[sf,scale=0.95,ttscale=.9]{libertine}
\usepackage[cmintegrals,cmbraces]{newtxmath}
\usepackage{ebgaramond-maths}
\makeatletter %^^A am unrhyw reswm, dyw ebgaramond ddim yn diffinio'r rhai hyn
  \DeclareRobustCommand{\oldstylefam}{\fontfamily{EBGaramond-\ebgaramond@figurealign OsF}\selectfont}
  \DeclareRobustCommand{\liningfam}{\fontfamily{EBGaramond-\ebgaramond@figurealign LF}\selectfont}
  \DeclareRobustCommand{\tabularfam}{\fontfamily{EBGaramond-T\ebgaramond@figurestyle}\selectfont}
  \DeclareRobustCommand{\proportionalfam}{\fontfamily{EBGaramond-\ebgaramond@figurestyle}\selectfont}
\makeatother
\EnableCrossrefs
\CodelineIndex
\RecordChanges
% \OnlyDescription
\DoNotIndex{\verb,\ProvidesPackageSVN,\NeedsTeXFormat,\ProcessKeyOptions}
\usepackage{babel}
\usepackage{fancyhdr}
\usepackage[plain]{fancyref}
\usepackage{array}
\usepackage{enumitem}
\makeatother
\usepackage{booktabs}
\usepackage{xcolor}
\usepackage{xurl}
% \urlstyle{tt}
\urlstyle{sf}
\usepackage{microtype}
\usepackage[a4paper,headheight=14pt,marginparwidth=45mm,hmarginratio=4:1,vscale=.8,hscale=.7,verbose]{geometry}	% use 14pt for 11pt text, 15pt for 12pt text
% addaswyd o chronos.tex
\newlength\tewadjust
\newlength\modiwldoctemplgth
\AddToHook {begindocument/after}[.]
{%
  \setlength\tewadjust{\marginparwidth+\marginparsep-\paperwidth+\textwidth+\oddsidemargin+1in}%
  \setlength\modiwldoctemplgth{0pt}%
}
\usepackage{csquotes}
\MakeAutoQuote{‘}{’}
\MakeAutoQuote*{“}{”}
\usepackage{caption}
\DeclareCaptionFont{lf}{\lstyle}
\captionsetup[table]{labelfont=lf}
% sicrhau hyperindex=false: llwytho CYN bookmark
\usepackage{hypdoc}% ateb Ulrike Fischer: https://tex.stackexchange.com/a/695555/
\usepackage{bookmark}
\hypersetup{%
  colorlinks=true,
  citecolor={moss},
  extension=pdf,
  linkcolor={strawberry},
  linktocpage=true,
  pdfcreator={TeX},
  pdfproducer={pdfeTeX},
  urlcolor={blueberry}%
}
\NewDocElement[%
  idxtype=lig.,
  idxgroup=ligatures,
  printtype=\textit{lig.},
]{Lig}{ligature}
\NewDocElement[%
  idxtype=sw.,
  idxgroup=swashes,
  printtype=\textit{sw.},
]{Sw}{swash}
\NewDocElement[%
  idxtype=pkg.,
  idxgroup=,
  printtype=\textit{pkg.},
]{Pkg}{package}
\NewDocElement[%
  idxtype=fd.,
  idxgroup=fds,
  printtype=\textit{fd.},
]{Fd}{fdfile}
\NewDocElement[%
  idxtype=drv.,
  idxgroup=drivers,
  printtype=\textit{drv.},
]{Drv}{driver}
\NewDocElement[%
  idxtype=map,
  idxgroup=map files,
  printtype=\textit{map},
]{Map}{mapfile}
\NewDocElement[%
  idxtype=etx,
  idxgroup=font encodings,
  printtype=\textit{etx},
]{Etx}{encoding}
\NewDocumentCommand \val { m }
{%
  {\ttfamily =\,\meta{#1}}%
}
\ExplSyntaxOn
\NewDocumentCommand \vals { m }
{
  {
    \ttfamily = \, 
    \clist_use:nn { #1 } { \textbar }
  }
} 
\keys_define:nn { ebgaramond-maths / doc }
{
  unknown .code:n = {
    \cs_if_free:cT { \l_keys_key_str }
    {
      \tl_gset:cn { \l_keys_key_str } { #1 }
    }
  },
}
\NewDocumentCommand \ebgaramond-mathsdocset { +m }
{
  \keys_set:nn { ebgaramond-maths / doc } { #1 }
}
\ExplSyntaxOff
\ebgaramond-mathsdocset{%
  bug={\href{https://codeberg.org/cfr/ebgaramond-maths/issues}{\textsc{bugtracker}}},
  codeberg={\href{https://codeberg.org/cfr/ebgaramond-maths}{\textsc{codeberg}}},
  github={\href{https://github.com/cfr42/ebgaramond-maths}{\textsc{github}}},
  ctan={\href{https://ctan.org/}{\textsc{ctan}}},
}
% \usepackage{cleveref}
\newcommand*{\gust}{\textsc{Gust}}
\newcommand*{\lpack}[1]{\textsf{#1}}
\newcommand*{\fgroup}[1]{\textsf{#1}}
\newcommand*{\fname}[1]{\textsf{#1}}
\newcommand*{\file}[1]{\texttt{#1}}
\title{\filebase}
\author{Clea F. Rees\thanks{%
    Bug tracker:
    \href{https://codeberg.org/cfr/ebgaramond-maths/issues}{\url{codeberg.org/cfr/ebgaramond-maths/issues}}
    \textbar{} Code:
    \href{https://codeberg.org/cfr/ebgaramond-maths}{\url{codeberg.org/cfr/ebgaramond-maths}}
    \textbar{} Mirror:
    \href{https://github.com/cfr42/ebgaramond-maths}{\url{github.com/cfr42/ebgaramond-maths}}% 
}}
% \date{\fileversion~\filetoday}
\date{\fileversion~\filedate}
\pagestyle{fancy}
\fancyhf{}
\fancyhf[lh]{\itshape\filebase}
% ^^A \fancyhf[rh]{\itshape\filetoday}
\fancyhf[rh]{\itshape\fileversion}
% ^^A \fancyhf[ch]{\itshape Support for Latin Modern}
% ^^A\fancyhf[lf]{\itshape\fileversion}
\fancyhf[cf]{\itshape--- \thepage~/~\lastpage{} ---}
\ExplSyntaxOn
\hook_gput_code:nnn {shipout/lastpage} {.}
{
  \property_record:nn {t:lastpage}{abspage,page,pagenum}
}
\cs_new_protected_nopar:Npn \lastpage 
{
  \property_ref:nn {t:lastpage}{page}
}
\cs_new_eq:NN \OrigMakePrivateLetters \MakePrivateLetters
\ExplSyntaxOff
\definecolor{strawberry}{rgb}{1.000,0.000,0.502}
\definecolor{blueberry}{rgb}{0.000,0.000,1.000}
\definecolor{moss}{rgb}{0.000,0.502,0.251}
\begin{document}
\let\MakePrivateLetters\MyMakePrivateLetters
\DocInput{\filename}
\end{document}
%</driver>
% \fi
%
% \changes{v1.0}{???}{First public release.}
% ^^A lua will replace v0.0 and 0000/00/00 when tagging
% \changes{v1.3}{2025/02/02}{Update for changes to type1 fonts.
%   Switch to docstrip.}
% 
% \maketitle\thispagestyle{empty}
% \pdfinfo{%
% 	/Creator		(TeX)
% 	/Producer		(pdfTeX)
% 	/Author			(Clea F. Rees)
% 	/Title			(ebgaramond-maths)
% 	/Subject		(TeX)
% 	/Keywords		(TeX, LaTeX, font, fonts, tex, latex, ebgaramond-maths, EBGaramond, ebgaramond, EBGaramond, maths, mathematics, Clea, Rees)}
% \pdfcatalog{%
% 	/URL				()
% 	/PageMode	/UseOutlines}	
% \setlength{\parindent}{0pt}
% \setlength{\parskip}{0.5em}
% 
% \begin{abstract}
%   \liningfam
%   \noindent\lpack{ebgaramond-maths} provides some \LaTeX\ support for the use of EBGaramond in mathematics.
%   It requires \lpack{ebgaramond} and uses the postscript fonts provided by that package\footnote{This version corresponds to the 2019--05--04 version of \lpack{ebgaramond}.}.
%   The package essentially consists of the files generated by an answer to a question at \url{http://tex.stackexchange.com/q/152038/} and is a response to a follow-up request by the author of that question.
% \end{abstract}
%
% \tableofcontents
% \section{Introduction}\label{sec:intro}
% ^^A BEGIN sec:intro
% There is not really much to document.
% To use EBGaramond in mathematics, you just load the package:
% \begin{quote}
% \iffalse
%<*verb>
% \fi
\begin{verbatim}
\usepackage[cmintegrals,cmbraces]{newtxmath}
\usepackage{ebgaramond-maths}
\end{verbatim}
% \iffalse
%</verb>
% \fi
% \end{quote}
% Note that this will also load \lpack{ebgaramond} which will set your default serif font to \liningnums{EBGaramond}.
% If for any reason you do not want this, note two things.
% First, your document will be a typographic abomination.
% Second, you can easily create the abomination of your choice by changing the default serif family \emph{after} loading \lpack{ebgaramond-maths}.
% ^^A END sec:intro
% 
% \section{Details}\label{sec:manylion}
% ^^A  BEGIN sec:manylion
% 
% The package includes \verb|.tfm| and \verb|.map| files which define EBGaramond-Maths, a new \TeX\ font for \LaTeX.
% This font uses \liningnums{EBGaramond-Italic} with an OML encoding.
% However, not all characters in this encoding are available (see \fref{tab:dim}).
% Note that this is a limitation of the font itself and not of this package.
% 
% \begin{table}
%   \centering
%   \caption{Symbols missing from EBGaramond}\label{tab:dim}
%   \begin{tabular}{llll}
% 	\toprule
% 	\string\leftharpoonup & \string\triangleright & \string\flat & \string\smile\\
% 	\string\leftharpoondown & \string\triangleleft &  \string\natural & \string\frown\\
% 	\string\rightharpoonup & \string\star & \string\sharp & \string\vec\\
% 	\string\rightharpoondown & \string\partial & & \string\t\\
% 	\bottomrule
%   \end{tabular}
% \end{table}
% 
% 
% \lpack{ebgaramond-maths} uses this new font together with support files from \lpack{ebgaramond} to set up support for mathematics as follows:
% \begin{itemize}
%   \item EBGaramond-Maths (medium weight) is used for \verb|letters| (standard and bold);
%   \item EBGaramond-LF (medium weight, upright shape) is used for \verb|operators| (standard and bold);
%   \item EBGaramond-LF (medium weight, swash shape) is used for the calligraphic alphabet, \verb|\mathcal| (medium weight);
%   \item \verb|\mathrm|, \verb|\mathbf| and \verb|\mathit| should work as expected and use EBGaramond-LF (medium weight, upright or italic shape as appropriate).
% \end{itemize}
% 
% Note that \liningnums{EBGaramond} does not include a bold weight by design.
% Following the designer's intentions, this package, like \lpack{ebgaramond}, substitutes the medium weight for bold where required.
% ^^A  END sec:manylion
% 
% \section{Method}\label{sec:dull}
% ^^A  BEGIN sec:dull
% 
% If you just wish to use the package, you do not need to read this section.
% It explains how to create the font support files used by the package given that you have \lpack{ebgaramond} installed.
% It assumes that you are using TeX Live on GNU/Linux or another Unix-like system such as OS X.
% 
% \subsection{Variables}\label{subsec:var}
% ^^A  BEGIN subsec:var
% Make a working directory somewhere which I'll call \verb|${work}|.
% The only requirement is that you have permission to write there and a (very small) amount of space.
% (It goes without saying that this should not be done as root.)
% 
% In the instructions which follow \verb|${texmain}| is your main, current \verb|texmf| directory.
% On my system, that's \verb|/usr/local/texlive/YYYY| (where \verb|YYYY| is the latest version of TeX Live installed) or \verb|/usr/local/texlive/current|.
% ^^A  END subsec:var
% 
% \subsection{Working environment}\label{subsec:amgylchedd}
% 
% ^^A  BEGIN subsec:amgylchedd
% Change to \verb|${work}|.
% From now on, I assume that all commands are executed in this directory and that all created files are saved there.
% 
% Create the following symbolic links in your working directory\footnote{%^^A
%   On Windows, you will need to copy the file instead.%^^A
% }:
% 
% \begin{quote}
% \iffalse
%<*verb>
% \fi
\begin{verbatim}
ln -s ${texmain}/texmf-dist/tex/fontinst/mathetx/oml.etx \
  ${texmain}/texmf-dist/fonts/opentype/public/ebgaramond/\
  EBGaramond-Italic.otf ./
\end{verbatim}
% \iffalse
%</verb>
% \fi
% \end{quote}
% ^^A  END subsec:amgylchedd
% 
% \subsection{Create a preliminary encoding file}\label{subsec:creu-enc}
% ^^A  BEGIN subsec: creu-enc
% This is not the encoding file \TeX\ needs but it will form the basis for that file.
% 
% First, run \verb|fontinst| in interactive mode.
% (That is, just type \verb|fontinst| at the command.)
% At the prompt:
% \begin{quote}
% \iffalse
%<*verb>
% \fi
\begin{verbatim}
\input finstmsc.sty
\etxtoenc{oml}{oml-ebgaramond}
\bye
\end{verbatim}
% \iffalse
%</verb>
% \fi
% \end{quote}
% 
% This will produce \verb|oml-ebgaramond.enc| which should be lightly modified before feeding to \verb|otftotfm|\footnote{%^^A
%   On Windows, you will need to substitute an equivalent command or edit the file by hand.%^^A
% }:
% 
% \begin{quote}
% \iffalse
%<*verb>
% \fi
\begin{verbatim}
sed -i -e 's/TeXMathItalicEncoding/\
  EBGaramondTeXMathItalicEncoding/g' \
  -e 's/oldstyle//' oml-ebgaramond.enc
\end{verbatim}
% \iffalse
%</verb>
% \fi
% \end{quote}
% 
% This ensures that the encoding has a distinctive (and hopefully unique) name.
% ^^A  END subsec: creu-enc
% 
% \subsection{Generate the \TeX\ font}\label{subsec:tfm}
% ^^A  BEGIN subsec:tfm
% \lpack{ebgaramond} already provides the \liningnums{type1} files needed so there is no need to regenerate those.
% All that is required is to generate a suitable \verb|.tfm|:
% 
% \begin{quote}
% \iffalse
%<*verb>
% \fi
\begin{verbatim}
otftotfm -e oml-ebgaramond.enc EBGaramond-Italic.otf \
  > EBGaramond-Maths.map
\end{verbatim}
% \iffalse
%</verb>
% \fi
% \end{quote}
% 
% This will create both the \verb|.tfm| file and the \verb|.map| file fragment \TeX\ needs to use the font.
% It will also create a new encoding file with what will almost certainly be a very weird name.
% This is the encoding file \TeX\ will use, as specified in the \verb|.map| file fragment.
% The temporary encoding \verb|oml-ebgaramond.enc| can now be deleted as it is no longer required.
% ^^A  END subsec:tfm
% 
% ^^A  END sec:dull
% 
% \MaybeStop{%
% \PrintChanges
% \PrintIndex
% }
% 
% \section{Implementation}
%
% You do not need to read the remainder of this document in order to install or use the package.
%
% \iffalse
%<*sty>
% \fi
% \changes{v0.0}{2025-02-06}{Support for bold. Follow \texttt{ebgaramond} series settings.}
%    \begin{macrocode}
\NeedsTeXFormat{LaTeX2e}
\RequirePackage{svn-prov}
\ProvidesPackageSVN[\filebase.sty]{$Id: ebgaramond-maths.dtx 10780 2025-02-07 08:15:26Z cfrees $}[v1.3 \revinfo]
\DefineFileInfoSVN
%    \end{macrocode}
% \iffalse
% ^^A Paid â defnyddio \GetFileInfoSVN*/\GetFileInfoSVN{} yn y fan hon!!
% \fi
%    \begin{macrocode}

\RequirePackage{ebgaramond}

\DeclareSymbolFont{letters}   {OML}   {EBGaramond-Maths} {\mdseries@rm} {it}
\DeclareSymbolFont{operators} {OT1}   {EBGaramond-LF}    {\mdseries@rm} {n}

\SetSymbolFont{letters}   {bold}  {OML} {EBGaramond-Maths} {\bfseries@rm} {it}
\SetSymbolFont{operators} {bold}  {OT1} {EBGaramond-LF}    {\bfseries@rm} {n}

\DeclareFontSubstitution{OML}{EBGaramond-Maths}{\mdseries@rm}{it}
\DeclareFontSubstitution{OT1}{EBGaramond-LF}{\mdseries@rm}{n}

\SetMathAlphabet{\mathbf}   {normal}  {OT1} {EBGaramond-LF} {\bfseries@rm} {n}
\SetMathAlphabet{\mathbf}   {bold}    {OT1} {EBGaramond-LF} {\bfseries@rm} {n}
\SetMathAlphabet{\mathit}   {normal}  {OT1} {EBGaramond-LF} {\mdseries@rm} {it}
\SetMathAlphabet{\mathit}   {bold}    {OT1} {EBGaramond-LF} {\bfseries@rm} {it}

\DeclareMathAlphabet{\mathcal} {OT1} {EBGaramond-LF} {\mdseries@rm} {sw}
\SetMathAlphabet{\mathcal}  {bold}    {OT1} {EBGaramond-LF} {\bfseries@rm} {sw}

%^^A The following symbols are missing and should give errors

\gdef\ebgaramond@maths@help{%^^A
  EBGaramond does not provide this symbol.\MessageBreak
  If you are using the recommended setup with newtxmath\MessageBreak
  you can use \string\re@DeclareMathSymbol{}{}{}{} to take it from another font.\MessageBreak
  For example, to take symbols from Computer Modern:\MessageBreak
  \expandafter\noexpand\string\DeclareSymbolFont{cmletters}{OML}{cmm} {m}{it}\MessageBreak
  Then a specific symbol, such as \string\leftharpoonup, can be defined as follows:\MessageBreak
  \expandafter\noexpand\string\re@DeclareMathSymbol{\string\leftharpoonup}{\mathrel}{cmletters}{"28}}

%^^A Warning based on David Carlisle's answer at http://tex.stackexchange.com/a/214524/
\newcommand*{\ebgaramond@maths@dim}{\leftharpoonup,\leftharpoondown,\rightharpoonup,\rightharpoondown,\triangleright,\triangleleft,\star,\partial,\flat,\natural,\sharp,\smile,\frown,\vec,\t}
\@for \xx:=\ebgaramond@maths@dim \do {%^^A
  \expandafter\edef\xx{\noexpand\PackageError{ebgaramond-maths}{No \expandafter\string\xx}{\ebgaramond@maths@help}}}

%    \end{macrocode}
% \iffalse
%</sty>
% \fi
% 
% \subsection{Font Definitions}\label{subsec:fds}
% 
% \iffalse
%<*fd>
% \fi
%    \begin{macrocode}
%^^A Filename: OMLEBGaramond-Maths.fd
%% Based on a file created using fontinst v1.928

%% THIS FILE SHOULD BE PUT IN A TEX INPUTS DIRECTORY

\ProvidesFileSVN{$Id: ebgaramond-maths.dtx 10780 2025-02-07 08:15:26Z cfrees $}[v1.3 \revinfo][OMLEBGaramond-Maths.fd: font definitions for OML/EBGaramond-Maths.]

\DeclareFontFamily{OML}{EBGaramond-Maths}{}

\DeclareFontShape{OML}{EBGaramond-Maths}{m}{it}{
	<-> EBGaramond-Italic--oml-ebgaramond
	}{}
\DeclareFontShape{OML}{EBGaramond-SemiBoldMaths}{sb}{it}{
	<-> EBGaramond-SemiBoldItalic--oml-ebgaramond
	}{}
\DeclareFontShape{OML}{EBGaramond-BoldMaths}{b}{it}{
	<-> EBGaramond-BoldItalic--oml-ebgaramond
	}{}
\DeclareFontShape{OML}{EBGaramond-ExtraBoldMaths}{eb}{it}{
	<-> EBGaramond-ExtraBoldItalic--oml-ebgaramond
	}{}
\DeclareFontShape{OML}{EBGaramond-MediumMaths}{medium}{it}{
	<-> EBGaramond-MediumItalic--oml-ebgaramond
	}{}

\DeclareFontShape{OML}{EBGaramond-Maths}{bx}{it}{<->ssub * EBGaramond-Maths/b/it}{}

%    \end{macrocode}
% \iffalse
%</fd>
% \fi
% \subsection{Sample}
% \iffalse
%<*ee>
% \fi
%    \begin{macrocode}
\documentclass{article}
\pdfmapfile{-EBGaramond-Maths}
\pdfmapfile{+EBGaramond-Maths}
\usepackage{ebgaramond}
\usepackage[cmintegrals,cmbraces]{newtxmath}
\usepackage{ebgaramond-maths}

\begin{document}
ABCDEFGHIJKLMNOPQRSTUVWXYZ

abcdefghijklmnopqrstuvwxyz

1234567890

$ABCDEFGHIJKLMNOPQRSTUVWXYZ$

$abcdefghijklmnopqrstuvwxyz$

$1234567890$

$\Gamma\varGamma\Delta\Lambda\varLambda\Xi\varXi\Pi\varPi\Sigma\varSigma\Upsilon\varUpsilon\Phi\varPhi\Psi\varPsi\Omega\varOmega$

$\alpha\beta\gamma\delta\epsilon\varepsilon\zeta\eta\theta\iota\kappa\lambda\mu\nu\xi\pi\rho\varrho\sigma\varsigma\tau\upsilon\phi\varphi\chi\psi\omega$

$\mathbf{ABCDEFGHIJKLMNOPQRSTUVWXYZ}$

$\mathbf{abcdefghijklmnopqrstuvwxyz}$

$\mathbf{0123456789}$

$\mathit{ABCDEFGHIJKLMNOPQRSTUVWXYZ}$

$\mathit{abcdefghijklmnopqrstuvwxyz}$

$\mathit{0123456789}$

$\mathrm{ABCDEFGHIJKLMNOPQRSTUVWXYZ}$

$\mathrm{abcdefghijklmnopqrstuvwxyz}$

$\mathrm{0123456789}$

$\mathcal{ABCDEFGHIJKLMNOPQRSTUVWXYZ}$

$\mathcal{abcdefghijklmnopqrstuvwxyz}$

$\mathcal{0123456789}$
%    \end{macrocode}
% \iffalse
%</ee>
% \fi
%\Finale
